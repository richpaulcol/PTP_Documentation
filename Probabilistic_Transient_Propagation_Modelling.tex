\documentclass[]{article}
 \usepackage{amsmath}
 \usepackage{amssymb}
% \usepackage{amsthm}
% \usepackage{textcomp}
 \parindent=0mm
 \parskip=2.5mm
% \parindent=0mm
% \renewcommand{\arraystretch}{1.35}
% \usepackage{cite}

%\usepackage[acronym,nonumberlist]{glossaries}
%\makeglossaries 
\usepackage{graphicx}
\graphicspath{{Figures/}}
\usepackage[small,bf,up]{caption}
\providecommand{\diff}[3]{\frac{d^{#3} #1}{d #2^{#3}}}
\providecommand{\pdiff}[3]{\frac{\partial^{#3} #1}{\partial #2^{#3}}}
\providecommand{\abs}[1]{\left \lvert#1\right \rvert}
\providecommand{\etal}[0]{\textit{et al. }}
\oddsidemargin 15mm                        % 1 inch + lefthand margin on odd numbered pages
\evensidemargin 15mm                       % 1 inch + lefthand margin on even numbered pages
\textwidth 145mm  

%\usepackage[round]{natbib}
%

\usepackage[square,numbers,comma,sort&compress]{natbib}
\usepackage{listings} 
%\bibliographystyle{plainnat}
\bibliographystyle{unsrt}
%\usepackage{nomencl}
%\nomlabelwidth=15mm
%\makeindex
%\makenomenclature

\date{\today}
\title{Probabilistic Transient Propagation (PTP), Background and Mathematics}

\author{Richard Collins}



\begin{document}
\maketitle

\begin{abstract}
 
\end{abstract}

\section{Transient Equations}
Here we are going to model the basic transient equations, 
\begin{equation}\label{TranEqns}
 \begin{split}
 \pdiff{H}{t}{} + \frac{a^2}{gA}\pdiff{Q}{x}{} &= 0 \\
\pdiff{H}{x}{} + \frac{1}{gA}\pdiff{Q}{t}{} + \frac{\lambda}{2 g DA^2}Q \abs{Q} &= 0
 \end{split}
\end{equation}

These are a pair of coupled non-linear hyperbolic partial differential equations that describe the change in head and flow along a pipe.
They need to be coupled with suitable, initial and boundary conditions to complete the solution.

\section{Method of Characteristics}
The standard solution technique for solving \eqref{TranEqns} numerical is to use the Method of Characteristics (MOC).  
The MOC converts the partial differential equations into ordinary differential equations along characteristic lines in the solution space, along which the information about changes flows.
To apply the MOC to \eqref{TranEqns} we first multiply the momentum equation by a currently unknown factor $K$, then add the two equations together:
\begin{equation}
 \pdiff{H}{t}{} + \frac{a^2}{gA}\pdiff{Q}{x}{} + K \pdiff{H}{x}{} + K \frac{1}{gA}\pdiff{Q}{t}{} + K \frac{\lambda}{2 g DA^2}Q \abs{Q} = 0
\end{equation}
without explicitly going into why, if we set $K$ to be the wavespeed $a$, then we get the following:
\begin{equation}
 \pdiff{H}{t}{} + a \pdiff{H}{x}{} + \frac{a}{gA}\left(\pdiff{Q}{t}{} + a \pdiff{Q}{x}{} \right) + \frac{a \lambda}{2 g DA^2}Q \abs{Q} = 0
\end{equation}
The total derivative of a function $f(x,t)$ w.r.t time is given by $\pdiff{f}{t}{} + \diff{x}{t}{}\pdiff{f}{x}{}$ so we can see that we have formed a series of total derivatives.
\begin{equation}
 \diff{H}{t}{} \pm \frac{a}{gA}\diff{Q}{t}{} \pm \frac{a \lambda}{2 g DA^2}Q \abs{Q} = 0
\end{equation}
this pair of equations is only valid along the lines when 
\begin{equation}
 \diff{x}{t}{} = \pm a
\end{equation}


NEEDS A BIT IN HERE ABOUT HOW TO GET TO THE NEXT BIT

\begin{figure}[htp]
 \centering
 \caption{Mesh discretisation for the MOC}
 \label{MOCmesh}
\end{figure}



\section{Kalman Filter}

\section{Extended Kalman Filter}
Note that the full transient equations are non-linear so the standard Kalman filter approach is unavailable. 
Instead we have to use the extended Kalman filter which uses the Jacobian of the evolution equations to approximate the evolution of the system covariance. 


Here we include the Extended Kalman Filter Equations. 

The basic equation relies on the ability to formulate the problem into a way which allows the next state of the system is only a function of the previous state, some control inputs (which we will not consider for now) and some uncertainty.
\begin{equation}
\begin{split}
 \mathbf{\hat{x}}_t &= f(\mathbf{\hat{x}_{t-1}}) + \mathbf{\hat{w}}_t \\
 \mathbf{\hat{z}}_t &= h(\mathbf{\hat{x}_t}) + \mathbf{\hat{v}}_t \\
\end{split}
\end{equation}

where $f(\cdot)$ and $h(\cdot)$ are some, potentially, nonlinear mappings of the evolution of the state and the measurements, and $w$ and $v$ are the zero-mean Gaussian noise vectors. 
The state evolution can be calculated in what ever non-linear way is most appropriate, i.e. by a matrix method, or any other technique. 
\begin{equation}
 \mathbf{\hat{x}}_t = f(\mathbf{\hat{x}_{t-1}})
\end{equation}
however the covariance matrices are calculated using:
\begin{equation}
 \mathbf{P}_t = \mathbf{F}_{k-1} \mathbf{P}_{t-1} \mathbf{F}_{k-1}^T + \mathbf{Q}_{t-1}
\end{equation}

where $\mathbf{F}$ is the Jacobian matrix of the function $f(\cdot)$
\begin{equation}
 \mathbf{F} = \pdiff{f}{\mathbf{\hat{x}}}{}
\end{equation}


\subsection{Calculation of the Jacobian}
To calculate the Jacobian of the evolution equation we first need to determine our state vectors and actually what our state evolution equations are:

\subsubsection{State Vector}
In this formulation the state vector is composed of the Heads and flows at each nodal position, however due to the non-linearity we also need to include the pipe diameter, friction factor (and possibly the pipe wave speed) on each pipe reach between nodes.  
If the number of Nodes in the model is $N$, the size of our state vector (not including wavespeed) will be $2N + 2(N-1) = 4N -2$, (or $5N-3$ if wave speed is included).



\end{document}
